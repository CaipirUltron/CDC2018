%%%%%%%%%%%%%%%%%%%%%%%%%%%%%%%%%%%%%%%%%%%%%%%%%%%%%%%%%%%%%%%%%%%%%%%%%%%%%%%%%%%%%%%%%%%%%%%%%%%%%%%%%
\begin{abstract}

The majority of works in line of sight (LOS) stabilization and tracking using inertially stabilized platforms (ISP) apply simple linear controllers to achieve the required performance. 
%
Commonly, linear models are employed to describe the relationship between torque and position of the ISP joints, such as a double integrator with an inertia gain.
%
%However, high-accuracy and/or fast motion applications may often require more involved control techniques.
%
However, these techniques do not provide ideal disturbance rejection or finite-time convergence, which are desired characteristics for these type of systems in the context of high-accuracy applications.

In this work, we propose a novel Sliding Mode Control (SMC) strategy 
%based on the Super-Twisting Algorithm (STA) 
for both stabilization and orientation tracking for a sensor in a 3-DOF ISP installed on a moving base.
%such as a vehicle.
%
Two distinct cases are considered: full state feedback and output feedback only.
%
In the latter case, a High-Order Sliding Mode observer (HOSMO) is proposed for the estimation of the ISP joint velocities.
%
In each case, two Super Twisting Controllers (STC) are employed in a cascade topology. 
%
The inner controller ideally rejects the \textit{dynamic} disturbances acting on the ISP joints, uncoupling the system into an ideal double integrator. 
%
The outer controller ensures orientation tracking in quaternion space, ideally rejecting all remaining kinematic disturbances.

Numerical simulations using Matlab show the efficiency and performance of the proposed controller and observer.

\end{abstract}

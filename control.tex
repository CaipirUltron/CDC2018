%%%%%%%%%%%%%%%%%%%%%%%%%%%%%%%%%%%%%%%%%%%%%%%%%%%%%%%%%%%%%%%%%%%%%%%%%%%%%%%%%%%%%%%%%%%%%%%%%%%%%%%%
\section{Super-Twisting Control}

\subsection{Super-Twisting Control with Full State Feedback}
\label{sec:STC_state_feedback}

In this section, a novel second-order sliding mode (SOSM) controller based on super-twisting algorithm (STC) will be developed for the stabilization and tracking of the LOS.

Consider \eqref{eq:HEADS_dynamics}, rewritten here as:
%
\begin{equation}
M_{qq} \, \ddot{q} + \tau_d = \tau_q \,,
\label{eq:inner_model}
\end{equation}
%
where $\tau_q \in \mathbb{R}^{3}$ is the vector of generalized forces acting on the robot joints, collocated with $\dot{q}$ and $\tau_d = C_{qq} \, \dot{q} + G_{q} + M_{qV} \, \dot{V}^{b}_{0b} + C_{qV} \, V^{b}_{0b} \in \mathbb{R}^{3}$ is a disturbance vector.
%
%Matrices $M_{qq}(q) \in \mathbb{R}^{3 \times 3}$ and $M_{qV}(q) \in \mathbb{R}^{3 \times 6}$ are mass matrices,
%%
%$C_{qq}(q,\dot{q},V^{b}_{0b}) \in \mathbb{R}^{3 \times 3}$ and $C^{b}_{qV}(q,\dot{q},V^{b}_{0b}) \in \mathbb{R}^{3 \times 6}$ are Coriolis matrices 
%%
%and $G^{b}_{q}(q,\eta_{b_2}) \in \mathbb{R}^{n}$ is the gravity vector.
%
The dynamic model \eqref{eq:inner_model} can be rewritten as:
%
\begin{align}
\dot{x}_1 &= x_2 \,, \nonumber \\
\dot{x}_2 &= M^{-1}_{qq}(x_1,\Pi)\,\tau_q + x_3(x_1,x_2,\Pi,t) \,,
\label{eq:SS_inner_model}
\end{align}
%
where the states $x_1 = q$, $x_2 = \dot{q}$ are the ISP joint angles and velocities and $x_3 = -M^{-1}_{qq}(x_1) \,\tau_d$ is a state-dependent disturbance.

\begin{remark}
Note that, under assumption of torque control $u(t) = \tau_q$, state-space model \eqref{eq:SS_inner_model} is a double-integrator with a nonlinear high-frequency gain and a matched disturbance $x_3 \in \mathbb{R}^{3}$.
\end{remark}

First, define the following \textit{sliding surface}:
%
\begin{equation}
s_x = x_2 - \int^t_0{w(\tau)} \, d\tau\,. 
\label{eq:sx_sliding}
\end{equation}
%
where $w(\tau) \in \mathbb{R}^3$ is an arbitrary signal.
%
Due to \eqref{eq:SS_inner_model}, the dynamics of the sliding variable $s_x$ is given by:
%
\begin{equation}
\dot{s}_x = \dot{x}_2 - w(t) = M^{-1}_{qq} \, u + x_3 - w(t)\,.
\label{eq:dsx_sliding}
\end{equation}

The {\it super-twisting control law} is defined as:
%
\begin{align}
S_t(s,A,B) &= A \aket{s}^{1/2} + B \int^{t}_{0}{sgn(s) d\tau} \,, 
\label{eq:st_ctrl}
\end{align}
%
where matrices $A, B > 0$.
%
Using the inner control law
%
\begin{align}
u(t) &= M_{qq}(x_1,\Pi) \, \left[ w(t) - S_t(s_x,\Lambda_1,\Lambda_2) \right] \,,
\label{eq:u_sliding}
\end{align}
%
where $\Lambda_1,\Lambda_2 > 0$, \eqref{eq:dsx_sliding} becomes:
%
\begin{align}
\dot{s}_x &= - \Lambda_1 \, \aket{s_x}^{1/2} + w_x \,, \nonumber \\
\dot{w}_x &= - \Lambda_2 \, \aket{s_x}^{0} + \dot{x}_3 \,,
\label{eq:STC_system}
\end{align}
%
which is identified as the STA and is finite-time stable for $\norm{\dot{x}_3} < L_x$.
%
According to \cite{Moreno2012}, the STA guarantees SOSM in finite-time, which means that after a time $t>T_1>0$, $s_x = \dot{s}_x = 0$. Therefore, $\dot{x}_2 = w(t) \,\, \forall t>T_1$, even in the presence of the bounded disturbance $x_3$.

Now, in a similar way than in \eqref{eq:SS_inner_model}, \eqref{eq:quaternion_propagation} and \eqref{eq:camera_orientation3} can be rewritten as:
%
\begin{align}
\dot{y}_{1} &= \frac{1}{2} \, \left[ \begin{array}{cc}
- y_{12}^\mathsf{T} \\
y_{11}\,\mathbf{I}_3 + \widehat{y}_{12}
\end{array} \right] \, y_2 \,, \\
%\dot{y}_{11} &= -0.5 \, y^\mathsf{T}_{12} \, y_{2}  \,, \nonumber \\
%\dot{y}_{12} &= 0.5 \, ( y_{11} \, \mathbf{I}_{3} + \widehat{y}_{12} ) \, y_{2} \,, \nonumber \\
\dot{y}_2 &= J^{c}_{0c}(x_1,\Pi_g) \, \dot{x}_2 + y_3(x_1,x_2,\Pi_g,t) \,.
\label{eq:outer_model}
\end{align}
%
where the state $y^\mathsf{T}_1 = \overline{r}^\mathsf{T}_c = \mat{\,\,y_{11} \,\,\, y^\mathsf{T}_{12}\,\,}$ is the vector representation of the camera orientation $r_c \in \mathbb{H}^*$, with $y_{11} = \eta_c$ and $y_{12} = \epsilon_c$ being the scalar and vector components. 
%
State $y_2 = \omega^c_{c}$ is the camera angular velocity, while $y_3 = \dot{J}^c_{0c}\,\dot{q} + \dot{\omega}^c_{0}$ is another state-dependent disturbance.
%
\begin{remark}
Note that the state-space model \eqref{eq:outer_model} is a double integrator with a nonlinear high-frequency gain and a matched disturbance $y_3$ with respect to a control input $\dot{x}_2$.
\end{remark}

This structure strongly suggests the use of a \textit{cascade controller} for both stabilization and tracking. 
%Inspired by the system dynamics and by the works of \cite{Mao2017,Chalanga2016}, the following control scheme is proposed.

Given an orientation reference $r_{c_d}(t) \in \mathbb{H}^*$ for the camera, it can be represented in vector form by $\overline{r}^\mathsf{T}_{c_d}(t) = y^\mathsf{T}_{1_d}(t) = \mat{\,\,y_{11_d}(t) \,\,\, y^\mathsf{T}_{12_d}(t) \,\,}$. 
%
The angular velocity of the camera is also given as $\omega^c_{c_d}(t) = y_{2_d}(t)$. 
%
The quaternion and angular velocity errors can be defined as:
%
\begin{align}
e_{c} &= r_{c_d}(t) \circ r^*_c \,, 
\label{eq:quaternion_errors1} \\
e_{\omega} &= y_{2_d}(t) - y_{2} \,.
\label{eq:quaternion_errors2}
\end{align}

Note that when $r_{c} = r_{c_d}(t)$, the orientation error \eqref{eq:quaternion_errors1} is zero.
%The state errors are defined as:
%%
%\begin{align}
%e_{c} &= y_{11} \, y_{12_d}(t) - y_{11_d}(t) \, y_{12} -\hat{y}_{12_d}(t) \, y_{12} \,, 
%\label{eq:quaternion_errors1} \\
%e_{\omega} &= y_{2_d}(t) - y_{2} \,.
%\label{eq:quaternion_errors2}
%\end{align}
%
Define another \textit{sliding surface} as
%
\begin{equation}
s_{y} = e_{\omega} + K_{c} \, \mathsf{Im}(e_{c}) \,, \quad K_{c} > 0 \,,
\label{eq:quaternion_sliding_surface}
\end{equation}
%
whose dynamics is given by
%
%\begin{align}
%\dot{s}_{y} = \dot{y}_{2_d} - \dot{y}_2 \,+\, &
%K_{c} \, ( y_{11} \, \dot{y}_{12_d} + \dot{y}_{11} \, y_{12_d} -\dot{y}_{11_d} \, y_{12}
%\label{eq:quaternion_sliding_dynamics} \\
%& \,\,\, - y_{11_d} \, \dot{y}_{12} - \dot{\widehat{y}}_{12_d} \, y_{12} - \widehat{y}_{12_d} \, \dot{y}_{12}) \,. \nonumber
%\end{align}
%
\begin{align}
\dot{s}_{y} &= \dot{y}_{2_d} - J^{c}_{0c}(x_1) \, \dot{x}_2 - y_3 + K_{c} \, \psi(y_1,y_2,r_{c_d}) \,, \label{eq:quaternion_sliding_dynamics2}
\end{align}
%
with $\psi(y_1,y_2,r_{c_d}) = y_{11} \, \dot{y}_{12_d} - 0.5 \, y^\mathsf{T}_{12} \, y_{2} \, y_{12_d} - \dot{y}_{11_d} \, y_{12} - \dot{\widehat{y}}_{12_d} \, y_{12} - 0.5 \, y_{11_d} \, ( y_{11} \, \mathbf{I}_{3} - \widehat{y}_{12} ) \, y_{2} - 0.5 \, \widehat{y}_{12_d} \, ( y_{11} \, \mathbf{I}_{3} - \widehat{y}_{12} ) \, y_{2}$ 
and the derivatives of the reference are 
$\dot{y}_{11_d} = -0.5 \, y^\mathsf{T}_{12} \, y_{2_d}$ 
and 
$\dot{y}_{12_d} = 0.5 ( y_{11} \, \mathbf{I}_{3} - \widehat{y}_{12} ) \, y_{2_d}$.
%
%\begin{align}
%\psi &= y_{11} \, \dot{y}_{12_d} - 0.5 \, y^\mathsf{T}_{12} \, y_{2} \, y_{12_d} - 
%\dot{y}_{11_d} \, y_{12} - \dot{\widehat{y}}_{12_d} \, y_{12} \nonumber \\
%& - 0.5 \, y_{11_d} \, ( y_{11} \, \mathbf{I}_{3} - \widehat{y}_{12} ) \, y_{2} - 0.5 \, \widehat{y}_{12_d} \, ( y_{11} \, \mathbf{I}_{3} - \widehat{y}_{12} ) \, y_{2} \,, \nonumber \\
%\dot{y}_{11_d} &= -0.5 \, y^\mathsf{T}_{12} \, y_{2_d} \,, \qquad \qquad
%\dot{y}_{12_d} = 0.5 ( y_{11} \, \mathbf{I}_{3} - \hat{y}_{12} ) \, y_{2_d} \,. \nonumber
%\end{align}

\begin{assumption}
\label{ass:1}
Here, suppose that the system \eqref{eq:STC_system} is already sliding after a finite time $T_1>0$. Therefore, $\dot{x}_2 = w(t)$. Under this assumption, \eqref{eq:quaternion_sliding_dynamics2} becomes 
$$\dot{s}_y = \dot{y}_{2_d}(t) - J^{c}_{0c}(x_1,\Pi_g) \, w(t) - y_3 + K_{c} \, \psi(y_1,y_2,r_{c_d})\,.$$
\end{assumption}

Under Assumption \ref{ass:1}, the {\it outer} control law:
%
\begin{align}
w(t) &\!=\! J^{c}_{0c}(x_1,\Pi)^{-1} \left[\dot{y}_{2_d}(t) \!+\! K_{c} \, \psi \!+\! S_t(s_y,\Lambda_3,\Lambda_4)\right] \,.
\label{eq:quaternion_w_sliding}
\end{align}
%
can be applied to \eqref{eq:quaternion_sliding_dynamics2}, yielding:
%
\begin{align}
\dot{s}_y &= - \Lambda_3 \, \aket{s_y}^{1/2} + w_y \,, \nonumber \\
\dot{w}_y &= - \Lambda_4 \, sign(s_y) - \dot{y}_3 \,,
\label{eq:quaternion_STC_outer_system}
\end{align}
%
once again guaranteeing finite-time stabilization after a time $T_2>0$ and under the assumption of $\norm{\dot{y}_3} < L_y$. 
%
It means that for all $t \ge T_2$, the system is sliding and therefore, it follows the nonlinear dynamics of the sliding variable \eqref{eq:quaternion_sliding_surface}, which is asymptotically stable according to \cite{Siciliano2009}.

\begin{theorem}

\end{theorem}

\begin{proof}

\end{proof}

%%%%%%%%%%%%%%%%%%%%%%%%%%%%%%%%%%%%%%%%%%%%%%%%%%%%%%%%%%%%%%%%%%%%%%%%%%%%%%%%%%%%%%%%%%%%%%%%%%%%%%%%
\subsection{HOSMO Observer for STC with Output Feedback}
\label{sec:STC_output_feedback}

If all system states $q,\dot{q} \in \mathbb{R}^3$ are not available, but only the joint angles $q$, 
a second order differentiator can be used to estimate $\dot{q}$.
%
It was introduced by Levant in his seminal work \rev{\cite{}}. This system is a third-order observer based on the theory of HOSMs. 

Defining the estimation error $e_{x_1} = x_1 - \hat{x}_1$, the HOSM observer for the second order nonlinear system \eqref{eq:SS_inner_model} is:
%
\begin{align}
\dot{\widehat{x}}_1 &= k_1 \, \aket{e_{x_1}}^{2/3} + \widehat{x}_2 \,, \nonumber \\
\dot{\widehat{x}}_2 &= k_2 \, \aket{e_{x_1}}^{1/3} + \widehat{x}_3 + M^{-1}_{qq}(x_1,\Pi) \, u \,, \nonumber \\
\dot{\widehat{x}}_3 &= k_3 \, sgn(e_{x_1}) \,.
\label{eq:x_HOSMO}
\end{align}

%Defining also the estimation errors $e_{x_2} = x_2 - \widehat{x}_2$, $e_{x_3} = x_3 - \widehat{x}_3$ and using \eqref{eq:SS_inner_model}, the error dynamics for \eqref{eq:x_HOSMO} is:
%%
%\begin{align}
%\dot{e}_{x_1} &= -k_1 \, \aket{e_{x_1}}^{2/3} + e_{x_2} \,, \nonumber \\
%\dot{e}_{x_2} &= -k_2 \, \aket{e_{x_1}}^{1/3} + e_{x_3} \,, \nonumber \\
%\dot{e}_{x_3} &= -k_3 \, sgn(e_{x_1}) + \dot{x}_3 \,,
%\label{eq:error_x_HOSMO}
%\end{align}
%
%Its state space representation is given by:
%%
%\begin{align}
%\dot{z}_1 &= k_1 \, \aket{z_1 - \sigma(t)}^{2/3} + z_2 \,, \nonumber \\
%\dot{z}_2 &= k_2 \, \aket{z_1 - \sigma(t)}^{1/3} + z_3 \,, \nonumber \\
%\dot{z}_3 &= k_3 \, sgn(z_1 - \sigma(t)) \,.
%\label{eq:x_HOSMO}
%\end{align}
%
Its already proven by literature \cite{Angulo2013} that after a finite time $T_{reach}>0$, the states of the second-order differentiator \eqref{eq:x_HOSMO} uniformly converge to $\widehat{x}_1 = x_1$, $\widehat{x}_2 = x_2$ and $\widehat{x}_3 = x_3$, given that $\norm{\dot{x}_3} < L_x$.

Defining the {\it modified} sliding surface
%
\begin{equation}
\widehat{s}_x = \widehat{x}_2 - \int^t_0{w(\tau)} \, d\tau \,,
\label{eq:sx_sliding_HOSMO}
\end{equation}
%
its dynamics is given by:
%
\begin{align}
\dot{\widehat{s}}_x &= 
%\dot{\widehat{x}}_2 - w(t) 
k_2 \, \aket{e_{x_1}}^{1/3} + \widehat{x}_3 + M^{-1}_{qq}(x_1,\Pi) \, u - w(t) \,.
\label{eq:dsx_sliding_HOSMO}
\end{align}

Modifying control law \eqref{eq:u_sliding} to
%
\begin{equation}
u(t) \!=\! M_{qq}(x_1,\Pi) \left[ w(t) \!-\! k_2 \, \aket{e_{x_1}}^{1/3} \!-\! S_t(\widehat{s}_x,\Lambda_1,\Lambda_2) \right],
\label{eq:u_sliding_HOSMO}
\end{equation}
%
the exact same STA than in \eqref{eq:STC_system} is obtained, but with $s_x$ replaced by the modified sliding variable $\widehat{s}_x$. 
%
%\begin{align}
%\dot{\hat{s}}_x &= - \Lambda_1 \, \aketext{\hat{s}_x}{1/2} + \hat{w}_x \,, \nonumber \\
%\dot{\hat{w}}_x &= - \Lambda_2 \, sign(\hat{s}_x) + \dot{\hat{x}}_3 \,,
%\label{eq:STC_system_HOSMO}
%\end{align}

\begin{remark}
Another HOSMO could be designed to estimate the camera angular velocity $\omega_c^c(t)$. However, usually the camera orientation $r_c(t)$ is obtained from an {\it Inertial Measurement Unit (IMU)}, a device that combines measurements of gyroscopes (which measure angular velocity) and magnetometers (which measure magnetic fields), providing an accurate estimate for $r_c(t)$. Therefore, trustworthy direct measurements of $\omega^c_c(t)$ are usually already available.
\end{remark}

\begin{theorem}

\end{theorem}

\begin{proof}

\end{proof}

%%%%%%%%%%%%%%%%%%%%%%%%%%%%%%%%%%%%%%%%%%%%%%%%%%%%%%%%%%%%%%%%%%%%%%%%%%%%%%%%%%%%%%%%%%%%%%%%%%%%%%%%
\section{Super-Twisting Control with Quaternion Feedback}
\label{sec:control}

In this section, a novel second-order sliding mode (SOSM) controller based on super-twisting algorithm (STA) will be developed for the stabilization and tracking of the ISP.
%
Two cases are considered: STC with full state feedback and STC with output feedback.

Consider \eqref{eq:HEADS_dynamics}, rewritten here as:
%
\begin{equation}
M_{qq} \, \ddot{q} + \tau_d = \tau_q \,,
\label{eq:dynamic_model}
\end{equation}
%
where $\tau_q \in \mathbb{R}^{3}$ is the vector of generalized forces acting on the robot joints, collocated with $\dot{q}$ and $\tau_d = C_{qq} \, \dot{q} + G_{q} + M_{qV} \, \dot{V}^{0}_{0} + C_{qV} \, V^{0}_{0} \in \mathbb{R}^{3}$ is a disturbance vector.
%
%Matrices $M_{qq}(q) \in \mathbb{R}^{3 \times 3}$ and $M_{qV}(q) \in \mathbb{R}^{3 \times 6}$ are mass matrices,
%%
%$C_{qq}(q,\dot{q},V^{b}_{0b}) \in \mathbb{R}^{3 \times 3}$ and $C^{b}_{qV}(q,\dot{q},V^{b}_{0b}) \in \mathbb{R}^{3 \times 6}$ are Coriolis matrices 
%%
%and $G^{b}_{q}(q,\eta_{b_2}) \in \mathbb{R}^{n}$ is the gravity vector.
%
The dynamic model \eqref{eq:dynamic_model} can be rewritten as:
%
\begin{align}
\dot{x}_1 &= x_2 \,, \nonumber \\
\dot{x}_2 &= M^{-1}_{qq}(x_1,\Pi)\,\tau_q + x_3(x_1,x_2,\Pi,t) \,,
\label{eq:inner_model}
\end{align}
%
where the states $x_1 = q$, $x_2 = \dot{q}$ are the ISP joint angles and velocities and $x_3 = -M^{-1}_{qq}(x_1) \,\tau_d$ is a state-dependent disturbance.

\begin{remark}
Note that, under assumption of torque control $u(t) = \tau_q$, state-space model \eqref{eq:inner_model} is a double-integrator with a nonlinear high-frequency gain and a matched disturbance $x_3 \in \mathbb{R}^{3}$.
\end{remark}

Now, in a similar way than in \eqref{eq:inner_model}, \eqref{eq:quaternion_propagation} and \eqref{eq:camera_orientation3} can be rewritten as:
%
\begin{align}
\dot{y}_{1} &= \frac{1}{2} \, \mathbf{h}_+(y_1) \, y_2 \,, \nonumber \\
%\mathbf{h}_+(y_1) = \left[ \begin{array}{cc}
%- y_{12}^\mathsf{T} \\
%y_{11}\,\mathbf{I}_3 + \widehat{y}_{12}
%\end{array} \right] 
%\dot{y}_{11} &= -0.5 \, y^\mathsf{T}_{12} \, y_{2}  \,, \nonumber \\
%\dot{y}_{12} &= 0.5 \, ( y_{11} \, \mathbf{I}_{3} + \widehat{y}_{12} ) \, y_{2} \,, \nonumber \\
\dot{y}_2 &= J^{c}_{0c}(x_1,\Pi_g) \, \dot{x}_2 + y_3(x_1,x_2,\Pi_g,t) \,.
\label{eq:outer_model}
\end{align}
%
where the state $y^\mathsf{T}_1 = \overline{r}^\mathsf{T}_c = \mat{\,\,y_{11} \,\,\, y^\mathsf{T}_{12}\,\,}$ is the vector representation of the camera orientation $r_c \in \mathbb{H}^*$, with $y_{11} = \eta_c$ and $y_{12} = \epsilon_c$ being the scalar and vector components. 
%
State $y_2 = \omega^c_{c}$ is the camera angular velocity, while $y_3 = \dot{J}^c_{0c}\,\dot{q} + \dot{\omega}^c_{0}$ is another state-dependent disturbance.
%
\begin{remark}
\label{remark2}
Note that the state-space model \eqref{eq:outer_model} is a double integrator with a nonlinear high-frequency gain and a matched disturbance $y_3$ with respect to a control input $\dot{x}_2$.
\end{remark}

This structure strongly suggests the use of a \textit{cascade controller} for both stabilization and tracking. 
An inner controller acts on $u(t)$ in \eqref{eq:inner_model} to control $\dot{x}_2$, providing dynamic stabilization 
for the system, while an outer tracking controller acts on $\dot{x}_2$ in \eqref{eq:outer_model}, controlling the camera orientation $y_1$.
%Inspired by the system dynamics and by the works of \cite{Mao2017,Chalanga2016}, the following control scheme is proposed.

Given an orientation reference $r_{c_d}(t) \in \mathbb{H}^*$ for the camera, it can be represented in vector form by $\overline{r}^\mathsf{T}_{c_d}(t) = y^\mathsf{T}_{1_d}(t) = \mat{\,\,y_{11_d}(t) \,\,\, y^\mathsf{T}_{12_d}(t) \,\,}$. 
%
The angular velocity of the camera is also given as $\omega^c_{c_d}(t) = y_{2_d}(t)$. 
%
The quaternion and angular velocity errors can be defined as:
%
\begin{align}
e_{c} &= r_{c_d}(t) \cdot r^*_c \,, 
\label{eq:quaternion_errors1} \\
e_{\omega} &= y_{2_d}(t) - y_{2} \,.
\label{eq:quaternion_errors2}
\end{align}
%
Note that when $r_{c} = r_{c_d}(t)$, the orientation error \eqref{eq:quaternion_errors1} is zero.  

%%%%%%%%%%%%%%%%%%%%%%%%%%%%%%%%%%%%%%%%%%%%%%%%%%%%%%%%%%%%%%%%%%%%%%%%%%%%%%%%%%%%%%%%%%%%%%%%%%%%%%%%
\subsection{Super-Twisting Control with Full State Feedback}
\label{sec:STC_state_feedback}

Suppose that both ISP states $x_1$ and $x_2$ are available.
%
The following theorem provides an stability analysis for the proposed sliding mode cascade controller.

%%%%%%%%%%%%%%%%%%%%%%%%%%%%%%%%%%%%%%%%%%%%%%%%%%%%%%%%%%%%%%%%%%%%%%%%%%%%%%%%%%%%%%%%%%%%%%%%%%%%%%%%
\begin{theorem}[{\it Cascade STC with Full State Feedback}]

Let \eqref{eq:inner_model} and \eqref{eq:outer_model} be the system dynamic and kinematic models.
%
Assume the following:
%
\begin{enumerate}[label=(\roman*)]
%
\item the ISP joint velocities and accelerations are uniformly norm-bounded; \label{assump:1}
%there are $\rho_1,\,\rho_2 > 0$ such that: $\norm{\dot{q}} \leq \rho_1$, $\norm{\ddot{q}} \leq \rho_2$; 
%
\item 
the zero, first and second order time-derivatives of the vehicle velocity twists are uniformly norm-bounded; \label{assump:2}
%there are $\epsilon_1,\,\epsilon_2 > 0$ such that: $\norm{V^b_{0b}} \leq \epsilon_1$, $\norm{V^b_{0b}} \leq \epsilon_2$.
%
\end{enumerate}

Defining the {\it super-twisting control} expression
%
\begin{align}
S_t(s,A,B) &= A \aket{s}^{1/2} + B \int^{t}_{0}{sgn(s) d\tau} \,, \nonumber
\end{align}
%
with matrices $A, B > 0$, the super-twisting-based controllers can be defined as follows.
%
The outer sliding surface is
%
\begin{align}
s_{y} &= e_{\omega} + K_{c} \, \mathsf{Im}(e_{c}) \,, \quad K_{c} > 0 \,,
\label{eq:quaternion_sliding_surface}
\end{align}
%
where $K_c > 0$. The corresponding outer control law is
%
\begin{align}
w(t) &= \widehat{J}^{c}_{0c}(x_1)^{-1} \left[\dot{y}_{2_d}(t) \!+\! K_{c} \, \psi \!+\! S_t(s_y,\Lambda_3,\Lambda_4)\right] \,.
\label{eq:w_sliding}
\end{align}
%
where $\widehat{J}^{c}_{0c}(x_1) = J^{c}_{0c}(x_1,\widehat{\Pi}_g)$ and $\psi$ is a function of $y_1$, $y_2$ and $r_{c_d}$. 
%Matrices $\Lambda_3,\Lambda_4$ are positive-definite.
%
The inner sliding surface is defined as
%
\begin{equation}
s_x = x_2 - \int^t_0{w(\tau)} \, d\tau\,,
\label{eq:sx_sliding}
\end{equation}
%
and the corresponding inner control law is
%
\begin{align}
u(t) &= \widehat{M}_{qq}(x_1) \, \left[ w(t) - S_t(s_x,\Lambda_1,\Lambda_2) \right] \,,
\label{eq:u_sliding}
\end{align}
%
where $\widehat{M}_{qq}(x_1) = M_{qq}(x_1,\widehat{\Pi}_g,\widehat{\Pi}_d)$.
%
Then, control laws \eqref{eq:u_sliding} and \eqref{eq:w_sliding} ensure finite-time exact convergence 
of the sliding variables $s_x$ and $s_y$ as defined in \eqref{eq:sx_sliding} and 
\eqref{eq:quaternion_sliding_surface}. Furthermore, the quaternion and angular velocity errors 		$e_c$, 
$e_\omega$ are asymptotically stable under the dynamics of $s_y = 0$. 
%
%\begin{equation}
%e_{\omega} + K_{c} \, \mathsf{Im}(e_{c}) = 0 \,.
%\label{eq:quaternion_sliding_condition}
%\end{equation}

\end{theorem}
%%%%%%%%%%%%%%%%%%%%%%%%%%%%%%%%%%%%%%%%%%%%%%%%%%%%%%%%%%%%%%%%%%%%%%%%%%%%%%%%%%%%%%%%%%%%%%%%%%%%%%
\begin{proof}
%
Using \eqref{eq:inner_model} and \assref{remark2}, the dynamics of the sliding variable $s_x$ is given by:
%
\begin{align}
\dot{s}_x = \dot{x}_2 - w(t) = M^{-1}_{qq} \, u(t) + x_3 - w(t) \,.
\label{eq:dsx_sliding}
\end{align}
%
Substituting \eqref{eq:u_sliding} into \eqref{eq:dsx_sliding}, it becomes:
%
\begin{align}
\dot{s}_x &= - ( \mathbf{I}_3 - M_{qq}^{-1} \, \Delta M_{qq}  )\,S_t(s_x,\Lambda_1,\Lambda_2) + x_3 \,, 
\label{eq:dsx_sliding2}
\end{align}
%
where $\Delta M_{qq} = M_{qq} - \widehat{M}_{qq}$. Using \eqref{eq:dyn_regressor}, 
$\Delta M_{qq} \, S_t = Y^*_q \, \widetilde{\Pi}_d + \Delta Y^*_q \, \widehat{\Pi}_d$, with $\Delta Y^*_q = Y^*_q - \widehat{Y}^*_q$, where $Y^*_q = Y_q(x_1,0,S_t(s_x,\Lambda_1,\Lambda_2),0,0,0,0,\Pi_g)$ and $\widehat{Y}^*_q = Y_q(x_1,0,S_t(s_x,\Lambda_1,\Lambda_2),0,0,0,0,\widehat{\Pi}_g)$.
%
Then, it is possible to rewrite \eqref{eq:dsx_sliding2} as:
%
\begin{align}
\dot{s}_x &= - \Lambda_1 \, \aket{s_x}^{1/2} + w_x \,, \nonumber \\
\dot{w}_x &= - \Lambda_2 \, \aket{s_x}^{0} + d_x \,,
\label{eq:STC_system}
\end{align}
%
where $d_x = \nabla (M_{qq}^{-1} \, Y^*_q)\,\widetilde{\Pi}_d + \nabla (M_{qq}^{-1} \, \Delta Y^*_q)\,\widehat{\Pi}_d + \dot{x}_3$ is clearly dependent on the base motion and on the errors on the geometric and dynamic parameters. Here, the operator $\nabla$ denotes time differentiation.

Note that \eqref{eq:STC_system} is STA, and therefore is finite-time stable for bounded disturbances.
%
It is evident that, if the nominal parameters are well known, system \eqref{eq:dsx_sliding2} is only perturbed by $d_x \approx \dot{x}_3$.
%
%Suppose that the following inequalities hold:
Due to Assumptions \ref{assump:1} and \ref{assump:2}, the following inequalities hold:
%
\begin{align}
\norm{ \nabla (M_{qq}^{-1} \, Y^*_q)\,\widetilde{\Pi}_d } &< L_{x_1} \,, \\
\norm{ \nabla (M_{qq}^{-1} \, \Delta Y^*_q)\,\widehat{\Pi}_d } &< L_{x_2} \,, \\
\norm{ \dot{x}_3 } &< L_{x_3} \,.
\end{align}

Then, $\norm{d_x} < L_{x_1} + L_{x_2} + L_{x_3}$, and according to \cite{Moreno2012}, it is possible to chose $\Lambda_1$ and $\Lambda_2$ so that \eqref{eq:STC_system} achieves SOSM in finite-time. 
%
It means that after a time $T_1>0$, $s_x = \dot{s}_x = 0$ and due to \eqref{eq:dsx_sliding}, $\dot{x}_2 = w(t) \,\, \forall t>T_1$, even in the presence of the bounded disturbance $d_x$.

Next, using \eqref{eq:outer_model}, \eqref{eq:quaternion_errors1} and \eqref{eq:quaternion_errors2}, the dynamics of the outer sliding variable \eqref{eq:quaternion_sliding_surface} is given by
%
\begin{align}
\dot{s}_{y} &= \dot{y}_{2_d} - J^{c}_{0c}(x_1) \, \dot{x}_2 - y_3 + K_{c} \, \psi \,,
\label{eq:quaternion_sliding_dynamics2}
\end{align}
%
where $\psi(y_1,y_2,r_{c_d}) = y_{11} \, \dot{y}_{12_d} - 0.5 \, y^\mathsf{T}_{12} \, y_{2} \, y_{12_d} - \dot{y}_{11_d} \, y_{12} - \dot{\widehat{y}}_{12_d} \, y_{12} - 0.5 \, y_{11_d} \, ( y_{11} \, \mathbf{I}_{3} - \widehat{y}_{12} ) \, y_{2} - 0.5 \, \widehat{y}_{12_d} \, ( y_{11} \, \mathbf{I}_{3} - \widehat{y}_{12} ) \, y_{2}$, 
with $\dot{y}_{1_d} = \mathbf{h}_-(y_1) \, y_{2_d}$.
%$\dot{y}_{11_d} = -0.5 \, y^\mathsf{T}_{12} \, y_{2_d}$ and $\dot{y}_{12_d} = 0.5 ( y_{11} \, \mathbf{I}_{3} - \widehat{y}_{12} ) \, y_{2_d}$.

Since $\dot{x}_2 = \dot{s}_x + w(t)$, substituting \eqref{eq:w_sliding} into \eqref{eq:quaternion_sliding_dynamics2} yields:
%
\begin{align}
\dot{s}_y &= - \Lambda_3 \, \aket{s_y}^{1/2} + w_y \,, \nonumber \\
\dot{w}_y &= - \Lambda_4 \, \aket{s_y}^{0} + d_y \,,
\label{eq:quaternion_STC_outer_system}
\end{align}
%
where $d_y = - \dot{y}_3 - \nabla( J^c_{0c} \, \dot{s}_x ) - \nabla( W^*_\omega ) \widetilde{\Pi}_g$, with $W^*_\omega = W_\omega (x_1, w(t), 0)$, according to \eqref{eq:kin_regressor}.
%
%Again, suppose that the following inequalities hold:
Again, due to Assumptions \ref{assump:1} and \ref{assump:2}:
%
\begin{align}
\norm{ \nabla( J^c_{0c} \, \dot{s}_x ) } &< L_{y_1} \,, \label{eq:inequality_y1} \\
\norm{ \nabla( W^*_\omega ) \widetilde{\Pi}_g } &< L_{y_2} \,, \label{eq:inequality_y2} \\
\norm{ \dot{y}_3 } &< L_{y_3} \label{eq:inequality_y3} \,.
\end{align}
%
Note that \eqref{eq:inequality_y1} is reasonable, since $\ddot{s}_x$ is bounded, but constant 
$L_{y_1}$ clearly depends on the initial conditions of \eqref{eq:inner_model}.
%
Also, in \eqref{eq:inequality_y2}, $\nabla( W^*_\omega )$ depends on $x_1$, $x_2$, $w(t)$ and $\dot{w}(t)$, which are also bounded.
%
Then, $\norm{ d_y } < L_{y_1} + L_{y_2} + L_{y_3}$, again guaranteeing finite-time stabilization of \eqref{eq:quaternion_STC_outer_system} after a time $T_2>0$.
%
It means that for all $t \ge T_2$, the system is sliding and therefore, it follows the nonlinear dynamics of the sliding variable \eqref{eq:quaternion_sliding_surface}, which is asymptotically stable \cite{Siciliano2009}.
%
Therefore, the quaternion errors \eqref{eq:quaternion_errors1} and \eqref{eq:quaternion_errors2} tend to zero asymptotically after a time $max(T_1,T_2)$.

\end{proof}

%%%%%%%%%%%%%%%%%%%%%%%%%%%%%%%%%%%%%%%%%%%%%%%%%%%%%%%%%%%%%%%%%%%%%%%%%%%%%%%%%%%%%%%%%%%%%%%%%%%%%%%%
\subsection{Super-Twisting Control with HOSM Observer}
\label{sec:STC_output_feedback}

If state $x_2 \in \mathbb{R}^3$ is not available, an observer could be used to estimate the joint velocity state $x_2(t)$ using the measurements of $x_1(t)$.
%
Because of its desired characteristics such as finite-time exact convergence, sliding mode observers 
could be used for this purpose, such as the {\it super-twisting} observer (STO) \cite{Moreno2012}.
%
However, according to \cite{Chalanga2016}, it is not possible to achieve {\it second order sliding} ($s=\dot{s}=0$) using {\it continuous} control when STC is implemented based on ST observers. A proposed solution is to use STC with HOSM-based observers to achieve continuous control.

\begin{remark}
Two HOSMOs could be designed: one for the joint velocities $x_2(t)$, and other for the camera angular velocity $y_2(t)$. However, usually the camera orientation $y_1(t)$ is obtained from an {\it Inertial Measurement Unit (IMU)}, a device that combines measurements of gyroscopes (which measure angular velocity) and magnetometers (which measure magnetic fields), providing an accurate estimate for $y_1(t)$. Therefore, trustworthy direct measurements of $y_2(t)$ are usually already available.
\end{remark}

%The following theorem provides an stability analysis for the proposed STC and HOSM observer.

%%%%%%%%%%%%%%%%%%%%%%%%%%%%%%%%%%%%%%%%%%%%%%%%%%%%%%%%%%%%%%%%%%%%%%%%%%%%%%%%%%%%%%%%%%%%%%%%%%%%%%%%
\begin{theorem}[{\it Cascade STC with Output Feedback}]

Let \eqref{eq:inner_model} and \eqref{eq:outer_model} be the system dynamic and kinematic models.
%
Assume the following:
%
\begin{enumerate}[label=(\roman*)]
%
\item the ISP joint velocities and accelerations are uniformly norm-bounded; \label{hosm_assump:1}
%there are $\rho_1,\,\rho_2 > 0$ such that: $\norm{\dot{q}} \leq \rho_1$, $\norm{\ddot{q}} \leq \rho_2$;
%
\item the zero, first and second order time-derivatives of the vehicle velocity twists are uniformly norm-bounded; \label{hosm_assump:2}
%there are $\epsilon_1,\,\epsilon_2 > 0$ such that: $\norm{V^b_{0b}} \leq \epsilon_1$, $\norm{V^b_{0b}} \leq \epsilon_2$.
%
\end{enumerate}
%
Defining the estimation error $e_{x_1} = x_1 - \hat{x}_1$, the HOSM observer for $x_2$ is the third-order system:
%
\begin{align}
\dot{\widehat{x}}_1 &= K_1 \, \aket{e_{x_1}}^{2/3} + \widehat{x}_2 \,, \nonumber \\
\dot{\widehat{x}}_2 &= K_2 \, \aket{e_{x_1}}^{1/3} + \widehat{x}_3 + \widehat{M}^{-1}_{qq}(x_1) \, u \,, \nonumber \\
\dot{\widehat{x}}_3 &= K_3 \, \aket{e_{x_1}}^{0} \,.
\label{eq:x_HOSMO}
\end{align}
%
where $K_1$, $K_2$ and $K_3$ are positive-definite matrices.
%
The outer sliding variable and control law are defined in the same way as \eqref{eq:quaternion_sliding_surface} and \eqref{eq:w_sliding}.
%
The {\it modified} inner sliding variable is:
%
\begin{equation}
\widehat{s}_x = \widehat{x}_2 - \int^t_0{w(\tau)} \, d\tau \,,
\label{eq:sx_sliding_HOSMO}
\end{equation}
%
and the corresponding inner control law is:
%
\begin{equation}
u(t) \!=\! \widehat{M}_{qq}(x_1) \left[ w(t) \!-\! K_2 \, \aket{e_{x_1}}^{1/3} \!-\! S_t(\widehat{s}_x,\Lambda_1,\Lambda_2) \right].
\label{eq:u_sliding_HOSMO}
\end{equation}
%
Then, control laws \eqref{eq:u_sliding_HOSMO} and \eqref{eq:w_sliding} with observer \eqref{eq:x_HOSMO} ensure finite-time exact convergence of the sliding variables $s_x$ and $s_y$ as defined in \eqref{eq:sx_sliding_HOSMO} and \eqref{eq:quaternion_sliding_surface}, and of the estimation errors $e_{x_1}$, $e_{x_2} = x_2 - \widehat{x}_2$ and $e_{x_3} = x_3 - \widehat{x}_3$. Furthermore, the quaternion and angular velocity errors $e_c$, $e_\omega$ are asymptotically stable under the dynamics of $s_y = 0$.

\end{theorem}
%%%%%%%%%%%%%%%%%%%%%%%%%%%%%%%%%%%%%%%%%%%%%%%%%%%%%%%%%%%%%%%%%%%%%%%%%%%%%%%%%%%%%%%%%%%%%%%%%%%%%%%%
\begin{proof}

%HOSMOs are third-order systems introduced by Levant in his seminal work \cite{Levant2003}.
%
Using \eqref{eq:inner_model} and \eqref{eq:x_HOSMO}, the dynamics of the estimation errors is:
%
\begin{align}
\dot{e}_{x_1} &= -K_1 \, \aket{e_{x_1}}^{2/3} + e_{x_2} \,, \nonumber \\
\dot{e}_{x_2} &= -K_2 \, \aket{e_{x_1}}^{1/3} + e_{x_3} + (M^{-1}_{qq} - \widehat{M}^{-1}_{qq}) \, u \,, \nonumber \\
\dot{e}_{x_3} &= -K_3 \, \aket{e_{x_1}}^{0} + \dot{x}_3 \,.
\label{eq:error_x_HOSMO}
\end{align}
%
By using transformation $e_{x_4} = e_{x_3} + (M^{-1}_{qq} - \widehat{M}^{-1}_{qq}) \, u$, it is possible to rewrite \eqref{eq:error_x_HOSMO} as:
%
%
\begin{align}
\dot{e}_{x_1} &= -K_1 \, \aket{e_{x_1}}^{2/3} + e_{x_2} \,, \nonumber \\
\dot{e}_{x_2} &= -K_2 \, \aket{e_{x_1}}^{1/3} + e_{x_4} \,, \nonumber \\
\dot{e}_{x_4} &= -K_3 \, \aket{e_{x_1}}^{0} + d_{e} \,.
\label{eq:error_x_HOSMO2}
\end{align}
%
where $d_e = \dot{x}_3 + (M^{-1}_{qq} - \widehat{M}^{-1}_{qq}) \, \dot{u} + \nabla (M^{-1}_{qq} - \widehat{M}^{-1}_{qq}) \, u$. 
%
Due to Assumption \ref{hosm_assump:1} and \eqref{eq:u_sliding_HOSMO}, two constants $L_{e_1}, L_{e_2} > 0$ exist, such that:
%
\begin{align}
\norm{ (M^{-1}_{qq} - \widehat{M}^{-1}_{qq}) \, \dot{u} } &< L_{e_1} \,, \\
\norm{ \nabla (M^{-1}_{qq} - \widehat{M}^{-1}_{qq}) \, u } &< L_{e_2} \,.
\end{align}
%
Also, by Assumption \ref{hosm_assump:2}, $\norm{ \dot{x}_3 } < L_{x_3}$ also holds.
Then, $\norm{d_e} < L_{e_1} + L_{e_2} + L_{x_3}$, and therefore the disturbance $d_e$ is norm-bounded. According to \cite{Moreno2012_2}, it is possible to chose $K_1$, $K_2$ and $K_3$ so that the states on \eqref{eq:error_x_HOSMO2} are finite-time stable.
%
\begin{remark}
Since $M^{-1}_{qq} - \widehat{M}^{-1}_{qq} \ne 0$ due to parametric uncertainty, the estimation error $e_{x_3}$ is expected to be norm-bounded only. Therefore, $x_3 = \widehat{x}_3 + \mathbb{\beta}(\widetilde{\Pi})$, where $\mathbb{\beta}(\widetilde{\Pi})$ is a small residue dependent on the parametric uncertainty.
\end{remark}

The dynamics of the modified sliding variable is given by:
%
\begin{align}
\dot{\widehat{s}}_x &= K_2 \, \aket{e_{x_1}}^{1/3} + \widehat{x}_3 + \widehat{M}^{-1}_{qq}(x_1) \, u(t) - w(t) \,.
%\dot{\widehat{x}}_2 - w(t) 
\label{eq:dsx_sliding_HOSMO}
\end{align}
%
Using the {\it continuous} control law \eqref{eq:u_sliding_HOSMO}, yields:
%
\begin{align}
\dot{\widehat{s}}_x &= - \Lambda_1 \, \aket{\widehat{s}_x}^{1/2} + \widehat{w}_x \,, \nonumber \\
\dot{\widehat{w}}_x &= - \Lambda_2 \, \aket{\widehat{s}_x}^{0} + K_3 \aket{e_{x_1}}^{0} \,.
\label{eq:STC_HOSMO_system}
\end{align}
%
Since the disturbance $K_3 \aket{e_{x_1}}^{0}$ is obviously norm-bounded, the STA \eqref{eq:STC_HOSMO_system} is finite-time stable. Therefore, after a finite time $\bar{T}_1 > 0$, $\dot{x}_2 = w(t)$.

To prove the stability of the outer controller, a similar procedure is performed.
Since $\dot{x}_2 = \dot{\widehat{s}}_x + \dot{e}_{x_2} + w(t)$, substituting \eqref{eq:w_sliding} into \eqref{eq:quaternion_sliding_dynamics2} yields:
%
\begin{align}
\dot{s}_y &= - \Lambda_3 \, \aket{s_y}^{1/2} + w_y \,, \nonumber \\
\dot{w}_y &= - \Lambda_4 \, \aket{s_y}^{0} + \bar{d}_y \,,
\label{eq:quaternion_STC_HOSMO_outer_system}
\end{align}
%
where $\bar{d}_y = - \dot{y}_3 - \nabla( J^c_{0c} \, \dot{\widehat{s}}_x ) + \nabla ( J^c_{0c} \dot{e}_{x_2}) - \nabla( W^*_\omega ) \widetilde{\Pi}_g$.
%with $W^*_\omega = W_\omega (x_1, w(t), 0)$, according to \eqref{eq:kin_regressor}.
%
Again, due to Assumptions \ref{assump:1} and \ref{assump:2}, \eqref{eq:error_x_HOSMO2} and \eqref{eq:STC_HOSMO_system}, two positive constants $\bar{L}_{y_1}$, $\bar{L}_{y_2}$ exist, such that:
%
\begin{align}
\norm{ \nabla( J^c_{0c} \, \dot{\widehat{s}}_x ) } &< \bar{L}_{y_1} \,, \label{eq:HOSMO_inequality_y1} \\
\norm{ \nabla ( J^c_{0c} \dot{e}_{x_2}) } &< \bar{L}_{y_2} \,, \label{eq:HOSMO_inequality_y2}
\end{align}
%
Then, $\norm{ \bar{d}_y } < \bar{L}_{y_1} + \bar{L}_{y_2} + L_{y_2} + L_{y_3}$, again guaranteeing finite-time stabilization of \eqref{eq:quaternion_STC_HOSMO_outer_system} after a time $\bar{T}_2>0$.
%
%It means that for all $t \ge \bar{T}_2$, the system is sliding and therefore, it follows the nonlinear dynamics of the sliding variable \eqref{eq:quaternion_sliding_surface}, which is asymptotically stable \cite{Siciliano2009}.
%
Therefore, the quaternion errors \eqref{eq:quaternion_errors1} and \eqref{eq:quaternion_errors2} tend to zero asymptotically after a time $max(\bar{T}_1,\bar{T}_2)$.

\end{proof}
%%%%%%%%%%%%%%%%%%%%%%%%%%%%%%%%%%%%%%%%%%%%%%%%%%%%%%%%%%%%%%%%%%%%%%%%%%%%%%%%%%%%%%%%%%%%%%%%%%%%%%%%

